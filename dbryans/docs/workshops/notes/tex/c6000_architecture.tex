\documentclass{article}
\usepackage{color}
\usepackage{amsmath}
\usepackage{hyperref}
\usepackage{listings}
\hypersetup{
   colorlinks,
   citecolor=black,
   filecolor=black,
   linkcolor=black,
   urlcolor=black,
   linktocpage
 }
%\hypersetup{linktocpage}
\author{Barath Ramesh}
\title{C6000 Architecture Specific Topics}
\begin{document}
\maketitle
\tableofcontents
\newpage
\section{Intro to C6000 Architecture}
Solve Digital signal procssing
problem. ADC-->DSP(Algorithm)-->DAC. Heart of DSP algo is a MAC (sum
of products). Solve this problem efficiently, most of the DSP algo
should run faster.
\section{C6x CPU Architecture}
Load store architecture, read values in section A or B. Exectuion unit
(.M1, .M2, .L1, .L2). M-->Multiply, .L add(logical) together called
MACs. Supports 8 MACs/cycle (8x8 and 16x16).D1 .D2
load/store, .S1 .S2 shifts and other logical operations. 8 independent
functions (8 independent functional units) or * instructions per
cycle. Thus while MMACs speed math intensive algorithms, flexibility
of 8 independent functional units allows comiler to perform other
types fo processing. C6x CPU can dispatch up to eight parallel
instructions each cycle. All C6x instruction are conditional allowing
efficient hardware pipelining.
\section{C6000 C compiler optimization}
What does "Optimal Mean?"
\begin{itemize}
\item When my processing keeps up with my I/O (real time)
\item When my algo acheives theoritical minumum (purple patch)
\item After I have applied all known optmimization techniques
\end{itemize}
Real-time vs. CPU Min
\begin{itemize}
\item Typically, meeting real-time only requires setting a few compiler
  options (easy)
\item Acheiving "CPU Min" often requires extensive knowledge
  of the architecture (harder, requires more time).
\end{itemize}
\section{Optimization - Intro}
"Optimization is a continuous process of refinement in which code
being optimized executes faster and takes fewer cycles, until a
specific objective is acheived (real time execution). 
\begin{itemize}
\item Learn as many optimizations as possible and try them all if necessary
\end{itemize}
\section{Debug vs Optmized mode}
Turn on the optimizer for better performance. See Table
\ref{table:benchmarking} for few benchmarks with and without optimizations.
\begin{table}
\caption{Benchmarking}
\centering
\begin{tabular}{|c | c | c|}
\hline \hline
Alog & FIR(256, 64) & DOTP (256-term)\\
\hline \hline
Debug (no opt, -g) & 817K & 4109\\
\hline
Opt (-o3, no -g) & 18K & 42\\
\hline
Add'I pragmas & 7K & 42\\
\hline
(DSPLib) & 7K & 42\\
\hline
CPU Min & 4096 & 42\\
\hline \hline
\end{tabular}
\label{table:benchmarking}
\end{table}
\section{Build Configurations}
Two build configurations available debug and release (Set Active). Custom
optimizations can be made by clicking manage.
\begin{itemize}
\item Debig no opt, -g
\item Active -o3, no -g
\end{itemize}
Include paths are not copied from Debug to Active build
configuration. 
\section{Size Optimization}
Code size -ms works in conjunction with -o. Table \ref{table:codeSize}
\begin{table}
\caption{Minimizing code Space}
\centering
\begin{tabular}{|c| c| c| }
\hline \hline
-ms level & Performance & Code Size\\
\hline
none & 100\% & 0\%\\
\hline
-ms0 & 90\%  & 10\%\\
\hline
-ms1 & 60\%  & 40\%\\
\hline
-ms2 & 20\%  & 80\%\\
\hline
-ms3 & 0\%   & 100\%\\
\hline \hline
\end{tabular}
\label{table:codeSize}
\end{table}
\section{File and Function specific options}
Use compiler directives:
\begin{lstlisting}[language=C]
#pragma FUNCTION_OPTINS()
\end{lstlisting}
\section{Viewing Options in CCSv5.2}
Need to be done (how to set compiler optimization)
\section{Coding Guideline}
C, C++ source code --Compiler Optimizer--> 80\%-100\% efficiency with
minumum effort. ASM --Hand Optimize--> 100\% with high effort. Linear
ASM --Assembly Optimizer--> 95\%-100\% with medium effort.
\subsection{Basic C coding guidelines}
\begin{itemize}
\item Keep code complexity simple
\item No functions calls in tight loops
\item Keep llops relatively small
\item Look at the assembly file - SPLOOP, -k option to keep the
  assembly file
\end{itemize}
\section{Data types and Alignment}
\subsection{Supported Data types and format}
Data types supported is as shown in table \ref{table:dataTypes}
\begin{table}
\caption{C6000 Data Types} 
\centering
\begin{tabular}{|c| c| c|}
\hline \hline
Type & Bits  & Representation\\
\hline
char & 8   & ASCII\\
\hline
short & 16   & Binary 2's complement\\
\hline
int & 32   & Binary 2's complement\\
\hline
long & 40   & Binary 2's complement\\
\hline
float & 32   & IEEE 32-bit\\
\hline
double & 64 & IEEE 64-bit\\
\hline
long double & 64 & IEEE 64-bit\\
\hline
pointers & 32 & Binary\\
\hline \hline
\end{tabular}
\label{table:dataTypes}
\end{table}
\subsection{Data Alignment}
\begin{lstlisting}[language=C]
#pragma DATA_ALIGN(x, 4)
short z;
short x;
\end{lstlisting}
\section{Restricting memory dependence, Aliasing}

\subsection{Aliasing}
Two different ways of accessing a memory location:

\begin{lstlisting}[language=C]
int x;
int *p;

main(){

p = &x;

x=5;
*p = 8;
}
\end{lstlisting}
Such simple programs do not affect the compiler. To help the compiler
for s/w pipelining tell the compiler that they are two different
memory location (no aliasing) use \emph{restrict} key word.

\subsection{Alias Solution}
\begin{itemize}
\item Program level opttimization (-pm -03)
\item No bad Alisasing option (-mt), tell the compiler that there is
  no bad aliasing in my entire project
\item "Restrict" keyword (ANSI C) void fcn(short * in, short *
  restrict out)
\end{itemize}
\section{Accessing H/W features}
C code using intrinsics:
\begin{lstlisting}[language=C]
y = a * b;
y = \_mpyh(a, b);
\end{lstlisting}
In-Line Assembly:
\begin{lstlisting}[language=C]
asm(" MPYH A0, A1, A2");
\end{lstlisting}
Assembly code
\begin{lstlisting}[language=C]
MPYH A0, A1, A2
\end{lstlisting}
Intrinsics:
\begin{itemize}
\item Can use C variable names instead of register names
\item Are compatible with C environment
\item Adhere to C's function call syntax
\item Do not use in-line assembly
\end{itemize}
\section{Using Pragmas}
\begin{itemize}
\item \#pragma UNROLL(\# times to unroll)
\item \#pragma MUST\_ITERATE(min, max ,\%factor)
\end{itemize}
\end{document}

