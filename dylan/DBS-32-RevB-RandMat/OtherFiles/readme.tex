
%------------------------------------------------------------------------------%
%------------------------------ SETUP DOCUMENT --------------------------------%
%------------------------------------------------------------------------------%

\documentclass [letterpaper, 10pt, notitlepage, fleqn, oneside, 
  landscape] {article}

\makeatletter
\newcommand{\removelatexerror}{\let\@latex@error\@gobble}
\makeatother

%----------------------------- PACKAGE INCLUSION ------------------------------%
\usepackage[T1]{fontenc}
\usepackage{tabularx}
\usepackage{lipsum}
\usepackage[compact]{titlesec}

\usepackage[top=1.0cm, bottom=2.0cm, outer=1.5cm, inner=1.5cm,
  heightrounded, marginparwidth=4.0cm, marginparsep=0.5cm]{geometry}

%----------------------------- SET UP TITLE TEXT ------------------------------%
\usepackage{titlesec}
\usepackage{multicol}
\titleformat*{\section}{\Large \bf}
\titleformat*{\subsection}{\large \bf}
\titleformat*{\subsubsection}{\large}
\titleformat*{\paragraph}{\large}
\titleformat*{\subparagraph}{\large}

\usepackage{enumitem}
\setlist{nolistsep}

\newenvironment{Figure}
  {\par\medskip\noindent\minipage{\linewidth}}
  {\endminipage\par\medskip}

%------------------------------ PARAGRAPH SETUP -------------------------------%
\usepackage{setspace}
\onehalfspacing
\setlength{\parindent}{0pt}
\setlength{\parskip}{0.4ex}
\setlength{\columnsep}{2.5ex}
\expandafter\def\expandafter\normalsize\expandafter{%
    \normalsize
    \setlength\abovedisplayskip{1.5ex}
    \setlength\belowdisplayskip{1.5ex}
    \setlength\abovedisplayshortskip{1.5ex}
    \setlength\belowdisplayshortskip{1.5ex}
}

%-------------------------- DEFINE TITLE AND AUTHOR ---------------------------%
\title{\vspace{-5.0ex} \huge \bf \textsc{
    Using the Provided Benchmarks
    } \vspace{-1.5ex} }
\author{\large \bf \textsc{Dylan Rudolph - \today} }
\date{\vspace{-6.6ex}}


%------------------------------------------------------------------------------%
%------------------------------ START DOCUMENT --------------------------------%
%------------------------------------------------------------------------------%

\begin{document}

\setlength{\pdfpagewidth}{11in}
\setlength{\pdfpageheight}{8.5in}

\maketitle
\rule{\textwidth}{1pt}
\vspace{-0.50cm}
\begin{multicols}{2}

\section{Summary}

The goal of the files in this set is do automate the benchmark-gathering process
for any architecture which supports Python 2, has a fairly recent version of GCC, 
and runs a GNU/Linux or popular Unix-like operating system. Provided that the 
system is set up properly, only one command line call will be necessary to 
compile, run, and save the results of, all of the benchmarks.

\section{Dependencies} 

The following are necessary for proper operation:
\begin{itemize}[leftmargin=0.6cm]
  \item A recent version of GCC.
  \item A version of the OpenMP standard, which may come bundled with GCC.
  \item The FFTW library and development headers (Version 3), along with 
    the FFTW support for OpenMP.
  \item Some version of Python 2, where 2.7 is preferred.
  \item An operating system which supports GNU's Make functionality.
\end{itemize}

\section{Directory Map}

The main directory contains (1) the Python file \texttt{dylanbench.py} which 
does most of the automation process, (2) a set of Python scripts which call 
\texttt{dylanbench.py} to perform each benchmark, and (3) a shell script 
\texttt{do\_everything.sh} which calls each of the individual Python scripts.

The folder \texttt{Benchmarks} contains all of the relevant benchmarks. 
Inside of this folder there is a makefile which builds all of the benchmarks.
Inside of each of the individual benchmark folders there is also an 
individual makefile, along with the source code. If you do not want to compile 
all of the benchmarks, the individual makefiles can be called.

The folder \texttt{Savefiles} contains the \texttt{.bmark} files and a 
subfolder called \texttt{CSVs} which contains the output \texttt{.csv} 
files.

\section{Usage}

If all is well on the system, you need only call \texttt{sh do\_everything.sh} .

Alternatively, individual benchmarks can be compiled 
by calling \texttt{make} in the folder containing the benchmark. Also, 
individual benchmarks can be run by calling \texttt{python nameofbenchmark.py}.

\subsection{Notes}

\begin{itemize}[leftmargin=0.6cm]
  \item Make sure to change the number of threads for each of the OpenMP 
    benchmarks to match it for the platform.
  \item Ensure that the compiler optimization flags do not optimize out 
    blocks of code for the system of interest. 
\end{itemize}

\section{Benchmark Terminology}

Any benchmark appended with ``naive'' is a benchmark which has no compiler 
optimizations and is implemented in the most intuitive way. Any benchmark 
appended with ``OneThread'' is one which has been optimized (by compiler or 
algorithm) to make the best use of a single core on a host system. Finally, 
any benchmark labeled ``OMP'' is one which has been optimized and made parallel 
with OpenMP. 

\section{Save Files}

The \texttt{.bmark} format is specified by the \texttt{dylanbench.py} 
file, and contains most any important information about a benchmarking run. 
Some of the data from the \texttt{.bmark} format can be exported to a 
\texttt{.csv} file. This is done automatically by the accompanying 
Python scripts. Note that the \texttt{.csv} files contain less information 
than the \texttt{.bmark} files, so the \texttt{.bmark} should be considered 
the primary storage. Exporting from \texttt{.bmark} to \texttt{.csv} can be done 
by calling the relevant function from \texttt{dylanbench.py}.

\end{multicols}
\end{document}

